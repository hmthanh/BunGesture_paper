\documentclass[sigconf]{acmart}

\input{sections/_preamble.tex}
\input{sections/_template_args.tex}


%%
%% end of the preamble, start of the body of the document source.
\begin{document}

%%
%% The "title" command has an optional parameter,
%% allowing the author to define a "short title" to be used in page headers.
\title{BunGesture : Diffusion Gesture Generation Model for Vietnamese language}


%%
%% Authors.
%%
%% The "author" command and its associated commands are used to define
%% the authors and their affiliations.
%% Of note is the shared affiliation of the first two authors, and the
%% "authornote" and "authornotemark" commands
%% used to denote shared contribution to the research.
\author{Thanh Hoang-Minh}
\email{hmthanhgm@gmail.com}
\affiliation{
  \institution{Department of Information Technology, VNUHCM - University of Science}
  \streetaddress{227 Nguyen Van Cu Street, Ward 4}
  \city{Ho Chi Minh City}
  \state{District 5}
  \country{Vietnam}
  \postcode{70000}
}


\author{Ngoc Ly-Quoc}
\email{lqngoc@fit.hcmus.edu.vn}
\affiliation{
  \institution{Department of Information Technology, VNUHCM - University of Science}
  \streetaddress{227 Nguyen Van Cu Street, Ward 4}
  \city{Ho Chi Minh City}
  \state{District 5}
  \country{Vietnam}
  \postcode{70000}
}

%%
%% This command defines the author string for running heads.
%% By default, the full list of authors will be used in the page
%% headers. Often, this list is too long, and will overlap
%% other information printed in the page headers. This command allows
%% the author to define a more concise list
%% of authors' names for this purpose.
\renewcommand{\shortauthors}{Thanh, and Ngoc}

%%
%% The abstract is a short summary of the work to be presented in the
%% article.
\begin{abstract}
    Gesture generation is a pivotal aspect of multimodal communication systems, enhancing the naturalness and contextuality of human-machine interactions. While significant progress has been made in this field for various languages, gesture generation tailored specifically for Vietnamese remains underexplored. This paper introduces 
, a novel diffusion-based gesture generation model designed for the Vietnamese language. Unlike traditional approaches relying on English-centric models, bunGesture leverages Vietnamese-specific tools and techniques.

Key to bunGesture is its adoption of diffusion-based gesture generation, integrating advanced Vietnamese language models like PhoBERT and the VnCoreNLP toolkit. Additionally, bunGesture employs a fine-tuned Vietnamese version of WaveLM, trained on a custom Vietnamese audio dataset.

This work emphasizes the importance of language-specific adaptations in gesture generation systems and opens avenues for similar research in other languages.
\end{abstract}

%%
%% Article info.
%%
%% The code below is generated by the tool at http://dl.acm.org/ccs.cfm.
%% Please copy and paste the code instead of the example below.
%%
\begin{CCSXML}
<ccs2012>
   <concept>
       <concept_id>10010147.10010371.10010352</concept_id>
       <concept_desc>Computing methodologies~Animation</concept_desc>
       <concept_significance>500</concept_significance>
    </concept>
    <concept>
       <concept_id>10010147.10010178.10010179</concept_id>
       <concept_desc>Computing methodologies~Natural language processing</concept_desc>
       <concept_significance>300</concept_significance>
    </concept>
   <concept>
       <concept_id>10010147.10010257.10010293.10010294</concept_id>
       <concept_desc>Computing methodologies~Neural networks</concept_desc>
       <concept_significance>300</concept_significance>
    </concept>
 </ccs2012>
\end{CCSXML}

\ccsdesc[500]{Computing methodologies~Animation}
\ccsdesc[300]{Computing methodologies~Natural language processing}
\ccsdesc[300]{Computing methodologies~Neural networks}


%%
%% Keywords. The author(s) should pick words that accurately describe
%% the work being presented. Separate the keywords with commas.
\keywords{co-speech gesture synthesis, gesture generation, character animation, diffusion model, neural generative model, vietnamese}


%% A "teaser" image appears between the author and affiliation
%% information and the body of the document, and typically spans the
%% page.
\begin{teaserfigure}
  \includegraphics[width=\textwidth]{figures/teaser.pdf}
  \caption{Gesture results automatically synthesized by our system for a beat-rich TED talk clip. The red words represent beats, and the red arrows indicate the movements of corresponding beat gestures.}
  \Description{Enjoying the baseball game from the third-base
  seats. Ichiro Suzuki preparing to bat.}
  \label{fig:teaser}
\end{teaserfigure}

%%
%% This command processes the author and affiliation and title
%% information and builds the first part of the formatted document.
\maketitle

%%
%% Sections.
\section{Introduction}
\label{sec:introduction}

Every day, billions of people around the world look at RGB screens, and the output displayed on these screens is the result of various software systems. Therefore, the rendering of each pixel on the screen and the realistic simulation of images have been a focus of computer graphics scientists since the 1960s, particularly in the simulation of human figures or digital human.

Today, computer graphics technology can realistically simulate many complex objects such as water, roads, bread, and even human bodies and faces with incredible detail, down to individual hair strands, pimples, and eye textures. In 2015, using 3D scanning techniques \cite{metallo2015scanning} to capture all angles of the face and light reflection, researchers were able to recreate President Obama's face on a computer with high precision, making it almost indistinguishable from the real thing.

Artificial intelligence (AI) has shown remarkable results in recent years, not only in research but also in practical applications, such as ChatGPT and Midjourney, showcasing vertical and horizontal growth in various fields. Although computer graphics can construct highly realistic human faces, gesture generation has traditionally relied on Motion Capture from sensors, posing significant challenges in building an AI system that learns from data. Generating realistic beat gestures is challenging because gestural beats and verbal stresses are not strictly synchronized, and it's complicating for end-to-end learning models to capture the complex relationship between speech and gestures.


The main contributions of our work are as follows: 

% In summary, our main contributions in this paper are:
% \begin{itemize}
%     \item We present a novel rhythm- and semantics-aware co-speech gesture synthesis system that generates natural-looking gestures. To the best of our knowledge, this is the first neural system that explicitly models both the rhythmic and semantic relations between speech and gestures.
%     \item We develop a robust rhythm-based segmentation pipeline to ensure the temporal coherence between speech and gestures, which we find is crucial to achieving rhythmic gestures.
%     \item We devise an effective mechanism to relate the disentangled multi-level features of both speech and motion, which enables generating gestures with convincing semantics.
% \end{itemize}

\input{sections/2_related_work}
\input{sections/3_system_overview}
\input{sections/4_data_preparation}
\input{sections/5_gesture_generation}
\input{sections/6_co-speech_gesture_inference}
\input{sections/7_results}
\input{sections/8_conclusion}

%%
%% Acknowledgments.
% %%
% %% The acknowledgments section is defined using the "acks" environment
% %% (and NOT an unnumbered section). This ensures the proper
% %% identification of the section in the article metadata, and the
% %% consistent spelling of the heading.
% \begin{acks}
% To Robert, for the bagels and explaining CMYK and color spaces.
% \end{acks}

%%
%% The acknowledgments section is defined using the "acks" environment
%% (and NOT an unnumbered section). This ensures the proper
%% identification of the section in the article metadata, and the
%% consistent spelling of the heading.
% \begin{acks}
%     We thank the anonymous reviewers for their constructive comments.
%     This work was supported in part by NSFC
% \end{acks}
    

%%
%% The next two lines define the bibliography style to be used, and
%% the bibliography file.
\bibliographystyle{ACM-Reference-Format}
\bibliography{reference}

%%
%% Appendix.
\input{sections/9_appendix.tex}


\end{document}
\endinput
%%
%% End of file `sample-sigconf.tex'.
